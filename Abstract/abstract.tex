\chapter*{ABSTRACT}
\addcontentsline{toc}{chapter}{ABSTRACT}

Digital radiography systems are important diagnostic tools for modern medicine.  The images are produced when x-ray sensitive materials are coupled directly onto the sensing element of the detector panels.  As a result, the size of the detector panels is the same size as the x-ray image.  An alternative to the modern DR system is to image the x-ray phosphor screen with a lens onto a digital camera.  Potential advantages of this approach include rapid readout, flexible magnification and field of view depending on applications.  

We have evaluated lens-coupled DR systems for the task of signal detection by analyzing the covariance matrix of the images for three cases, using a perfect detector and lens, when images are affected by blurring due to the lens and screen, and for a signal embedded in a complex random background.  We compared the performance of lens-coupled DR systems using three types of digital cameras.  These include a scientific CCD, a scientific CMOS, and a prosumer DSLR camera.

We found that both the prosumer DSLR and the scientific CMOS have lower noise than the scientific CCD camera by looking at their noise power spectrum. We have built two portable low-cost DR systems, which were used in the field in Nepal and Utah. We have also constructed a lens-coupled CT system, which included a calibration routine and an iterative reconstruction algorithm written in CUDA.
