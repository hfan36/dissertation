\chapter{Reconstruction code}
The reconstruction code can be cloned from my public repository: \url{https://hfan36@bitbucket.org/hfan36/ct_reconstruction.git}
The main class that is used to perform image reconstruction is called: siddon\textunderscore recon.
This class has three main functions with image reconstruction as its main capability.  The function that can be used to perform CT image reconstruction is called a0\_RECON \textunderscore MLEM, which is an iterative MLEM algorithm.  There are also two other functions that can be used for simulation to calculate both forward projection images when given an input phantom volume and backward projection object volume when given a series of projection images.  These functions are called a1\_FORWARD\_PROJECTION and a1\_BACKWARD\_PROJECTION.  
All three functions requires a parameter file that sets the geometry of the CT system, detector size, object volume, and scan parameters as well as the file paths of the projection images for the reconstruction and backward projection function, or the object volume file name for the forward projection function.  An example of the parameter file is in the remote repository called ``MasterParameterFile.cfg'' and can be used to set to the desired values.  Once compiled, the program will first check for data files and consistencies in the parameter file before proceeding to do any calculations.  However it will not check for the memory size so make sure both the host and device side has enough RAM.  As a default setting, after each MLEM iteration the reconstruction function will write the calculated object to a file in the folder set in the parameter file, if the folder is left empty, then folder that house the projection images will be used.  A parameter file will also be written to the same location after each function has completed to show the values that was used.  This parameter file has the extension, ``.info'' to differentiate between the input parameter file that normally has the extension ``.cfg''. All outputs of the functions are written in binary in ``little-endian'' as floats.

\comment{
note somewhere that origin (0,0,0) is centered at the CT rotation axis, in the middle of the object volume
}