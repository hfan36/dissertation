\chapter{INTRODUCTION}
\label{chap:intro}
Digital radiography systems are important diagnostic tools for modern medicine.  They can be divided into two general groups using two different readout processes.  The first is based on storage phosphors where the x-ray image is first stored in an x-ray converter in a cassette form which later requires a separate optical readout process to record the image.  Typically, this separate readout process requires human intervention to transfer the storage phosphor cassette from the patient to the laser-scanning station.  Systems that acquire images using this method are commonly known as Computed Radiography (CR) systems, and they have been commercially available for almost two decades.  They are used for various applications and produce images with excellent image quality; however, they are not the focus of this dissertation.  For more information, see the review articles by Rowlands~\citep{Rowlands2002} and Kato~\citep{kato1994}, and the American Association of Physicists in Medicine (AAPM) Report No. 93.~\citep{AAPM93}.  

In the second group of radiography systems, the x-ray image is detected and read out by the same device without any human intervention.  These are commonly known as Digital Radiography (DR) systems and are the focus of this chapter.

\section{Digital Radiography (DR) detectors}
Modern x-ray digital radiography (DR) detectors were made possible by the considerable investment into developing active-matrix liquid-crystal flat-panel display (AMLCD) found in modern monitors and flat-screen TVs.  This technology created a way of manufacturing large-area integrated circuits called active-matrix arrays that enabled semiconductors, such as amorphous silicon, to be deposited across a large area on glass substrates.  The medical device community took advantage of this technology, which formed the basis of digital radiography detectors.  Sometimes called flat-panel detectors (FPD), they are built by coupling x-ray-sensitive materials with the active-matrix arrays that are created to store and read out the products of the x-ray interactions with sensitive materials, resulting in an image.  There are two general approaches to creating an x-ray detector, direct and indirect.  We will give a brief overview of the two approaches in the following section.

\subsection{Direct approach}
The terms direct and indirect refer to the outputs of initial x-ray interactions with the detection material rather than the design of the active-matrix arrays.  In the direct approach, an x-ray interaction with a photoconductor produces electron-hole pairs at the interaction site.  The detector signal is produced directly by collecting the electrons when an electric field is applied to the photoconductor, shown in Fig~\ref{fig:photoconductor_cross_section}
%  Currently, only amorphous-Selenium (a-Se) has been used in commercial x-ray detectors.
%
\begin{figure}[ht]
\centering
\includegraphics[scale = 1]{photoconductor_kasap.eps}
\caption{A cross-section of a photoconductor pixel\protect\footnotemark.  The charges are first generated by an incident x-ray photon, then collected onto a capacitor.  The collected charges will pass through a charge amplifier during readout when the gate line turns on the thin-film transistors (TFT) at each pixel.}
\label{fig:photoconductor_cross_section}
\end{figure}

\footnotetext{Reprinted from ``Recent advances in X-ray photoconductors for direct conversion X-ray image detectors'',6(3), need to finish.}

The x-ray sensitivity of detectors made using the direct approach depends on the photoconductor's ability to convert incident x-rays into collectible charges.  Several properties of the photoconductors affect their x-ray sensitivity.  First is the quantum efficiency of the photoconductor material.  The quantum efficiency refers to the absorbed fraction of incident radiation that is useful in creating electron-hole pairs.  The quantum efficiency for an x-ray of energy $E$ is given by $\eta_Q = 1 - exp[-\alpha (E, Z, \rho) T]$, where $T$ is the material's thickness, $\alpha$ is the linear attenuation coefficient of the material that is a function of the x-ray energy ($E$), the average atomic number of the material ($Z$), and the density of the material ($\rho$).  High quantum efficiency can be achieved by increasing the material's thickness, choosing a material with high $Z$ value, or density.  

\begin{figure}
\includegraphics[width = 0.9\linewidth]{photoelectron_augerelectron.pdf}
\caption{(a) A photoelectron is ejected from the K-shell by the absorption of an incident x-ray photon. (b) A characteristic x-ray photon is emitted when an electron from the L-shell is dropped down to fill the vacancy left by the photoelectron. (c) An Auger electron is ejected from its orbital shell when the energy released by the transitioning electron is absorbed~\citep{wiki_photoelectric}.}
\label{fig:photoelectric}
\end{figure}

A second property that affects the photoconductor's x-ray sensitivity is the generation of electron-hole pairs.  The predominant interaction of diagnostic x-ray with photoconductor medium is via the photoelectric effect, where the energy of an x-ray photon is transferred to the photoconductor's atom, and an electron is liberated from the atom's inner shell, shown in Fig.~\ref{fig:photoelectric}a.  The liberated electron is also called a photoelectron.  This event leaves behind a vacancy at the atom's inner shell, and is quickly filled by an electron from an outer orbital shell, which is in turn filled by an electron transitioning from a more distant shell.  This process of electrons cascading from one shell to another can release energies in the form of characteristic x-rays each with energy equal to the difference between the two transition shells, shown in Fig.~\ref{fig:photoelectric}b.  This cascading process can also release energies to eject Auger electrons, where the energy released by a cascading electron is used to eject another electron from its orbital shell, shown in Fig.~\ref{fig:photoelectric}c.  While the electrons cascade down to fill the vacancy created by the first photoelectron, the vacancy moves up through the outer shells of the atom to the photoconductor's valence band.  This vacancy is also referred to as a hole.  The characteristic x-rays released by the cascading electrons can also be absorbed within the photoconductor's medium to create more electron-hole pairs in a similar manner as the first photoelectron, albeit with electrons at higher orbital shells until all the energies have been absorbed.  The Auger electrons and the original photoelectron can also travel in the photoconductor's medium to create more electron-hole pairs by ionization until they lose all of their energies and come to a stop.  As a result, many electron-hole pairs are created by the absorption of one x-ray photon~\citep{Bushberg2002, Hajdok2006}.  The total charges generated from one absorbed photon is $e \, E / W_{\pm}$, where $e$ is the charge of an electron, $E$ is the energy of the incident photon, and $W_{\pm}$, is the energy required to create one electron-hole pair.  $W_{\pm}$ depends on the band-gap energy and in some cases such as a-Se, on the applied electric field~\citep{kasap2006}.
    
Another important property of the photoconductor is the mean distance traveled by a charge carrier.  In order to read out an image, the liberated charge carriers must be collected onto an external storage element before they are lost within the photoconductor material.  These charge carriers can be lost either by recombination of electrons with holes or they can be trapped at an unoccupied energy level between the conduction and valence band.  Electrodes that are placed on opposite ends of the material's surface create an electric field, which causes the free electrons and holes to drift in the opposite directions.  The mean distance traveled by a charge carrier before it is trapped or lost is called Schubweg.  This distance is given by $S = \mu \tau E$, which depends on the carrier's drift mobility ($\mu$), lifetime ($\tau$), and the applied electric field ($E$).  It is important that this distance is much longer than the thickness of the photoconductor material.  For example, at an applied field of 10 $\mathrm{V \mu m^{-1}}$, this distance is typically between 0.3 to 3 mm for an electron, and 6.5 to 65 mm for a hole in amorphous Selenium (a-Se).  The typical thickness of a-Se used for diagnostic imaging is between 0.2 to 1 mm~\citep{Rowlands2000}.

Problems that can affect x-ray detectors made with photoconductors are image lag and ghosting produced within the photoconductor material.  Image lag refers to the carried-over image produced from one exposure to the next.  This is caused by the trapped charges from one exposure becoming detrapped and read out in the subsequent image.  Ghosting refers to the trapped charges acting as recombination centers for the generated charges, which effectively reduce the lifetime of the charge carriers and the x-ray sensitivity.  These problems can be minimized by making sure the carrier's mean drift distance is larger than the material's thickness.

\begin{figure}[h]
	%
	\noindent \begin{minipage}[b]{0.4\textwidth}
		\centering
		\includegraphics[scale = 1]{photoconductor_conventional_circuit.pdf}
		\subcaption{}
	\end{minipage}
	%
	\hspace{2cm}
	%
	\begin{minipage}[b]{0.4\textwidth}
		\centering
		\includegraphics[scale = 1]{photoconductor_Zener_diode.pdf}
		\subcaption{}
	\end{minipage}
%	\vspace{0.5cm}
	
	\centering
	\begin{minipage}[b]{0.3\textwidth}
		\includegraphics[scale = 1]{photoconductor_dual_gate.pdf}
		\subcaption{}
	\end{minipage}
	\caption{Circuit for photoconductor based DR systems using (a) Conventional system, (b) Zener diode, and (c) dual-gate TFT.}
	\label{fig:photoconductor_circuit}
\end{figure}

Various x-ray photoconductor materials are used in commercial products, such as CdTe, CdZnTe, CdSe, PbO, $\mathrm{PbI_2}$, $\mathrm{HgI_2}$.  However these product applications typically involve small areas, less than 10 $\mathrm{cm^2}$.  Large area panels that are over 30 cm $\times$ 30 cm or greater, are typically made using amorphous Selenium (a-Se).  Due to its use as a photoreceptor for xerography~\citep{Mort1989}, and it ability to be deposited over a large area, a-Se is one of the most common photoconductor used commercially in direct digital radiography systems.

The biggest disadvantage of using a-Se is that it requires an internal field of approximately 10 $\mathrm{V \mu m^{-1}}$ to be maintained in order to activate the photoconductor layer.  So for a 500 $\mathrm{\mu m}$ layer, approximately 5,000 $V$ is required.  Both positive and negative bias voltage can be applied at the top electrode.  Shown in Fig~\ref{fig:photoconductor_circuit}a, if the applied bias voltage is positive, electrons are collected at the top electrode and holes are collected at the bottom charge collection electrode.  The capacitance of the a-Se layer is much smaller ($\sim$0.002 pF) than the pixel capacitor ($\sim$ 1 pF) so that the majority of the applied voltage is dropped across the photoconductor layer.  When the panel is left without scanning, dark or signal current will cause the potential on the pixel electrode to rise towards the applied bias voltage.  A voltage of $\sim$50 V can cause permanent damage to the thin-film transistor (TFT).  A simple method to protect the TFT is to use a negative bias at the top electrode so negative charges are collected at the pixel electrode.  Eventually sufficient charges accumulated at the storage capacitor cause the TFT to partially turn on and prevent the large potential from accumulating on the pixel (the gate voltage is not negative enough to turn off the TFT).  Shown in Fig~\ref{fig:photoconductor_circuit}b-c, other methods such as using a Zener-diode in parallel with the storage capacitor or modifying the TFT to incorporate a second gate will allow the potential accumulated on the pixel to drain away if it exceeds a predetermined safe design value.  However since these charges are drained out along the read-out lines, pixels sharing the same read-out lines as the over-exposed pixel can potentially contain corrupted information~\citep{Kasap2002, Rowlands2000}.  Another disadvantage of a-Se is that it has a relatively low atomic number, $Z$ = 34, which is not suitable for higher diagnostic x-ray energies ($\sim$60 keV).  As a result, a-Se is usually used in mammography devices operating at 20-30 kVp.
%

\subsection{Indirect approach}
In the indirect approach, detection materials such as phosphors or scintillators are placed in close contact with the active-matrix array.  An x-ray interaction in the detection material produces lower-energy photons typically in the visible range.  These lower-energy photons are then collected by a photosensitive element, such as a photodiode in each pixel, which in turn generates electrical charges.  These charges are then stored and read out by the active-matrix array to form an image.  The term indirect refers to the fact that x-ray interactions are detected indirectly using the electrical charges produced by the lower energy photons from the detection material rather than the electrical charges produced directly within the detection material.  

The most common materials used in flat-panel detectors that employ the indirect approach are $\mathrm{Gd_2O_2S:Tb}$ and $\mathrm{CsI:Tl}$.  Historically, powdered phosphors were deposited on plastic screens and were mainly used in x-ray imaging to expose photographic films; scintillators were grown as crystals and were used to detect high energy x- and gamma-rays~\citep{Nikl2006}.  In addition, phosphor screens and scintillators were prepared differently, though the fundamental physics behind both are identical.  

\begin{figure}[h]
\centering
\includegraphics[scale=1]{scintillator_band.pdf}
\caption{Energy band structure of (a) a semiconductor/photoconductor and (b) a scintillator/phosphor.}
\label{fig:scintillator_band}
\end{figure}

%We want phosphors and scintillators to absorb and convert as much incident radiation as possible and convert the energy to as many lower-energy photons as possible.  

The initial interaction between scintillators and photoconductors is identical, where photoelectric absorption takes place and many electron-hole pairs are created from absorption of an x-ray photon.  In an insulating crystal, the band-gap, $E_g$ between the valence and conduction band is large.  So, less electron-hole pairs are created, and the energy released when an electron-hole pair recombine is usually re-absorbed inside the material.  As a result, very little secondary photons are released.  In a scintillator or phosphor, we desire the radiative energy to escape the material without re-absorption and the conversion process to be more efficient.

In a scintillator or phosphor, the locations at the lattice defects and/or impurities introduce local discrete energy levels with excited and ground states between the forbidden gap.  When electrons and holes are created by the x-ray photon, the holes in valence band will quickly move up into the ground state created by the defects and/or impurities.  When an electron moving in the conduction band encounters an ionized site, it can drop down into the local energy level and de-excite into the ground state~\citep{Knoll2010}.  A more common process is via an exiton, where the electron in the conduction band is bound to a hole in the valence band.  This exiton can move freely in energy bands that are slightly below the conduction band.  When the exiton encounters an unoccupied energy level inside the forbidden gap, both the hole and electron are captured simultaneously.  This releases a photon with energy equal to the energy difference between the local excite and ground state.  Typically this energy is in the visible range (2-3 eV), and is smaller than the band-gap energy of the scintillator or phosphor.  These secondary photons cannot be re-absorbed to create more electron-hole pairs, so they are free to exit the material.  Since many electron-hole pairs are created by the absorption of one x-ray photon, and the energy of an x-ray photon is much larger than the energy of visible photon, many visible photons are created in the scintillator or phosphor by one x-ray photon.  For example, in $\mathrm{Gd_2O_2S:Tb}$ at $\sim$20$\%$ absorption efficiency, a 60 keV x-ray photon will produce approximately 5,000 green photons each with energy $\sim$2.4 eV.

\begin{figure}[ht]
	\includegraphics[scale = 1]{phosphorscreen_structure.eps}
	\caption[]{Cross section of a phosphor screen\footnotemark.}
\label{fig:phosphor_cross_section}
\end{figure}
\footnotetext{Reprinted from ``Radiological Imaging - The Theory of Image Formation, Detection, and Processing'',need to finish.}

\begin{figure}[ht]
\includegraphics[scale=1]{phosphorscreen_grains.eps}
\caption{The effects of (a) a thick phosphor layer, (b) a thin phosphor layer, and (c) an absorptive backing of x-ray screens on spatial resolution.}
\label{fig:phosphor_effects}
\end{figure}

The main issues with phosphor screens is that the spatial resolution is effected by optical scatter within the screen, which depends on screen thickness.  A thicker screen increases the probability of x-ray interactions, but lowers the spatial resolution.  When a photon exits a phosphor grain, it will scatter off the neighboring phosphor grains until it escapes the screen.  The final location where the photon is detected may not be the same as the initial x-ray interaction.  This spread of secondary photons lowers the spatial resolution.  Shown in Fig.~\ref{fig:phosphor_cross_section}, phosphor screens are typically made with several layers starting with a stiff plastic support to discourage severe bending.  The phosphor powders are sandwiched between a protective layer and a backing layer.  The backing layer can be made with an absorptive material to discourage optical diffusion, this increases spatial resolution at the cost of lowering the total number photons escaping the screen.  The backing layer can also be made with a white diffusive material to increase the light output at the cost of spatial resolution.  These effects can be seen in Fig~\ref{fig:phosphor_effects}.  Newer types of scintillators such as columnar $\mathrm{CsI:Tl}$ are grown as crystals in needle-like structures, which help to guide the emitted photons toward the exit surface.  These structures allow thicker scintillators to be made, which increase the probability of x-ray absorption while limit the spread of visible photons to within a few column structures, resulting in higher spatial resolution compared to traditional powdered phosphor screens.  This is shown in Fig.~\ref{fig:scintillators}.

\begin{figure}[ht]
\includegraphics[width = 7cm]{scintilators.jpg}
\caption{$\mathrm{Gd_2O_2S:Tb}$ phosphor and CsI scintillator viewed under SEM~\citep{scintillatorImage}.}
\label{fig:scintillators}
\end{figure}

\subsection{Readout arrays}

\begin{figure}[ht]
\includegraphics[width = 8cm]{active-matrix_array.png}
\caption[]{Schematic diagram of the main components of an active matrix array that are used to control the readout process\footnotemark~\citep{Fahrig2008}}
\label{fig:schematic_active-matrix_array}
\end{figure}
\footnotetext{pending permission from SPIE press}

Although the active-matrix arrays designed for detectors employing indirect and direct approach are slightly different, the readout schemes for both are exactly the same.  Unlike the readout process in charge-coupled devices (CCD), where the signals are transferred through pixels in columns and read out through a common output amplifier, here the signal in each pixel element is transferred directly to the readout amplifier.  Shown in Fig.~\ref{fig:schematic_active-matrix_array}, each row of the active-matrix array requires a separate gate line, and each column of the array pixels is connected to a separate data line each with its own charge amplifier.  During readout, the gate line in the first row of the array is turned on while all other rows are put in their off state.  This action turns on the thin-film transistors (TFTs) in the first row and allows signals from each pixel in the first row to flow through the data line.  Once all pixels have been read out in this row, the control switches the first row to the off state and turns on the second row, where the same procedure repeats again until all pixels in the flat-panel array have been read out.  

\subsection{Fill factor}
%
\begin{figure}[ht]
\includegraphics[scale=1]{photomicrograph.eps}
\caption[]{Photomicrographs showing a pixel on the active-matrix array with (a) a storage capacitor and (b) a photodiode\footnotemark~\citep{Rowlands2000}.}
\label{fig:pixel_fill_factor}
\end{figure}
\footnotetext{pending permission from SPIE press}
%
\begin{figure}[ht]
\includegraphics[scale=1]{mushroom_electrodes.eps}
\caption{Mushroom electrodes are used to increase the effective fill-factor of a pixel.}
\label{fig:mushroom_electrodes}
\end{figure}

One of the most important factors in flat-panel detector is the fill factor, which is the fraction of the pixel area that is sensitive to incoming signal.  In the indirect approach, this is the fraction of the photodiode area in the entire pixel that includes the photodiode, electrodes, the readout switch, and various control lines as shown in Fig.~\ref{fig:pixel_fill_factor}.  The fill factor in the direct approach can be much higher because the use of mushroom electrodes.  The mushroom electrodes extend over the top of the switching elements and bends the electric field, so the charges can drift away from dead zones and are collected onto the capacitor as seen in Fig.~\ref{fig:mushroom_electrodes}.

The design rule that is used to fabricate a particular active-matrix array governs many factors such as the thickness of the metallic lines and gaps between neighboring pixels, which are usually independent of the pixel sizes.  As a result, as pixels are made into smaller and smaller sizes, the fill factor will drop significantly, as seen in Fig.~\ref{fig:fill_factor}.

\begin{figure}[ht]
\includegraphics[width = 6cm]{Geo_fill_factor.PNG}
\caption[]{The geometric fill factor of pixels with different design rules and pixel sizes.  Here the gap size is the distance between electrodes/photodiodes\footnotemark ~\citep{Rowlands2000}.}
\label{fig:fill_factor}
\end{figure}
\footnotetext{pending permission from SPIE press}

\section{Lens-coupled x-ray detectors}
The use of a lens-coupled camera system as an x-ray detector has been proposed and rejected before.  The basic problem comes down to the collection efficiency of the photons produced by the phosphor or scintillator.  Enough photons from the phosphor screen need to be collected so the noise on the detector is limited by photon noise instead of inherent detector noise.  Although collection efficiency of flat-panel detectors depends on their fill-factor, the collection efficiency of lens-coupled x-ray detectors depends on lens' numerical aperture (NA).

The collection efficiency is the fraction of solid angle emitted by the source that is collected by the lens and focused onto the camera detector.  If we consider an on-axis source with a right circular cone, the solid angle can be calculated using
%
\begin{equation}
\label{eq:omega_lens}
\Omega = \int\limits_0^{2\pi} d \phi 
				\int\limits_0^{\Theta_{1/2}} sin \, \theta \ cos\, \theta \ d\theta = \pi \ sin^2 \, \Theta_{1/2},
\end{equation}
%
where $\theta$, $\phi$, and $\Theta_{1/2}$ are shown in Fig.X.  In a lens, the NA is equal to $n \, sin \theta$, where $n$ is the refractive index of the medium in front of the lens ($n = 1$ in air), and $\theta$ is the angle of the marginal ray with respect to the optical axis at the object.  So the solid angle collected by a lens in air is
%
\begin{equation}
\Omega_{lens} = \pi sin^2 \theta = \pi NA^2.
\end{equation}
%
The solid angle emitted by the source can be calculated similar to Eq.~\ref{eq:omega_lens}, and if the source is Lambertian, then $\Omega_{source} =  \pi$.  The collection efficiency of the lens is then
%
\begin{equation}
\label{eq:lens_collection_efficiency_all}
\eta = \frac{\Omega_{lens}}{\Omega_{source}} = sin^2 \Theta_{1/2} = NA^2,
\end{equation}
%
and this holds true for all lenses used in air.

The magnification of a lens, $m$, is given by
%
\begin{equation}
m = -q/p,
\label{eqn:magnification}
\end{equation}
%
where $p$ is the distance from the object to the lens' front principal plane (P), and $q$ is the distance from the lens' rear principal plane (P') to the image, shown in the top diagram in Fig.~\ref{fig:Abbe}.
%
\begin{figure}[h]
\includegraphics[width = 8cm]{AbbeSine.pdf}
\caption{The Abb$\acute{\mathrm{e}}$ sine condition.}
\label{fig:Abbe}
\end{figure}
%
If the lens is used in conditions that do not satisfy the paraxial approximation but are well corrected for spherical and coma aberrations, we can use the Abb$\acute{\mathrm{e}}$ sine condition to derive the collection efficiency.  Shown in the diagram in Fig.~\ref{fig:Abbe}, the Abb$\acute{\mathrm{e}}$ sine condition uses spherical surfaces rather than principal planes.  Here the distance $p$ and $q$ are the radius of the spherical surface rather than the distance of the object and image to the principal planes.  The condition states that $p \, sin \theta = q \, sin \theta'$ even when paraxial approximation ($sin \theta \approx tan \theta \approx \theta $) does not hold true, which might be the case when imaging a large object with a lens that has a large NA.  So the collection efficiency of the lens used under the Abb$\acute{\mathrm{e}}$ condition is
%
\begin{equation}
	\eta_{Abb\acute{e}} = m^2 sin^2 \theta',
\end{equation}
%
where $\theta'$ is the angle of the marginal ray in image space.

If the lens is used in conditions that satisfy the paraxial approximation, then we can use the F-number of the lens to calculate the collection efficiency.  The F-number describes the image-space cone of light for an object at infinity.  Under the paraxial approximation, this cone of light is approximately equal to~\citep{greivenkampfieldguide},
%
\begin{equation}
F \equiv \frac{f_E}{D_{EP}},
\end{equation}
%
where $D_{EP}$ is the diameter of the entrance pupil and $f_E$ is the effective focal length.  When two lenses are set to image at infinity and are mounted in a snout-to-snout fashion, then the diameter of the exit pupil is equal to the diameter of the entrance pupil.  The light cone between the two lenses is collimated and the numerical aperture of the lens-set is equal to the image forming cone of light described by the F-number.  The collection efficiency for the lens-set is then
%
\begin{equation}
\eta_{lens-set} = \frac{1}{F^2}.
\end{equation}
%
This set up can be used when unit magnification between the object and image is desired.

When a single lens is used for objects that are not at infinity, we can use the working F-number to describe the image-forming cone as,
%
\begin{equation}
F_{w} \approx (1 + \lvert m \rvert) \, F.
\end{equation}
%
The marginal ray angle in image space can then be related to the working F-number as,
%
\begin{equation}
sin \theta' = \frac{1}{2 F_{w}} = \frac{1}{2 F \, (1+ \lvert m \rvert)}.
\end{equation}
%
%
The collection efficiency of the lens under paraxial approximation is equal to
%
\begin{equation}
\eta_{paraxial} = \frac{m^2}{4 F^2(1 + |m|)^2}.
\end{equation}
%

While the phosphor screen must be at least the size of the object to be imaged, the lens must be able to capture the entire field of view and image the phosphor screen onto a CMOS or CCD detector with a limited size.  For a fixed field of view (FOV) and a small detector, we must move the lens and detector away from the object in order to fit the entire image of the object onto the sensor.  In order to increase the collection efficiency, which depends on the marginal ray angle, we can move to decrease $p$ and decrease $m$.  This means the detector size must be made larger.  Alternatively, we can increase the marginal ray angle by increase the aperture size of the lens.  This means using a lens with a high numerical aperture (NA).  In order to have both a large FOV and a high collection efficiency, we need both a large detector and a lens with high NA.  Commercial lenses with high NA (or low F-number) can be purchased at reasonable prices.  A F-1.4 DSLR lens is approximately $\$600$.  Detectors with large sensor size can be extremely expensive and difficult to manufacturer.  As a result, previous lens-coupled x-ray detector systems have been limited to small-scale imaging applications~\citep{lee2001, kim2005, tate2005, madden2006}.  

Recent sensor technology has improved tremendously, making it easier to purchase a camera with a large sensor size.  Current consumer-grade digital single-lens reflex (DSLR) cameras can be purchased with 36 mm $\times$ 24 mm detector size at around \$2000.  This improvement in sensor size allows us to decrease the distance between the lens and phosphor screen, therefore improving the photon collection efficiency while maintaining a large field of view.

In this dissertation, two x-ray imaging systems, a digital radiography (DR) system and a computed tomography (CT) system that were built using the concept of lens-coupled detector system are introduced in chapter~\ref{chap:design_construction}.  A method of evaluating x-ray CT detectors using an observer model is presented in chapter~\ref{chap:model_observer}.   The x-ray CT system presented is a fully functioning image system complete with calibration and reconstruction algorithms.  These algorithms are explained in chapters~\ref{chap:calibration} and~\ref{chap:reconstruction}.







































%
%
%
%
%
%\begin{figure}[ht]
%\includegraphics[scale=1.2]{lens_magnification3.eps}
%\caption{A comparison of (a) an imaging system with a light collection angle $\theta_1$ and $\lvert m_1 \rvert = q_1/p_1$, and (b) another imaging system with the same field of view but with a much larger light collection angle $\theta_2$ due to a lens with a larger NA and a larger detector size, where $ \lvert m_2 \rvert = q_2/p_2$ is larger than $\lvert m_1 \rvert$.}
%\label{fig:lens_magnification}
%\end{figure}
%
%While the phosphor screen must be at least the size of the object to be imaged, the lens must be able to capture the entire field of view and image the phosphor screen onto a CMOS or CCD detector with a limited size.  Shown in Fig.~\ref{fig:lens_magnification}, the magnification of a lens, $m$, is given by
%%
%\begin{equation}
%m = -q/p,
%\label{eqn:magnification}
%\end{equation}
%%
%where $p$ is the distance between the object to the lens' front principal plane (P), and $q$ is the distance from the lens' rear principal plane (P') to the image.  This same magnification is also the ratio between the size of the object and image as it is traced out by the chief ray.  For a fixed field of view (FOV) and a small detector, we must move the lens and detector away from the object in order to fit the entire image of the object onto the sensor.  In order to increase the collection efficiency, which depends on the marginal ray angle, we can move to decrease $p$.  This means the detector size must be larger.  Alternatively, we can increase the marginal ray angle by increase the aperture size of the lens.  This means using a lens with a high numerical aperture (NA).  In order to have a large FOV and high collection efficiency, we need both a large detector and a lens with a high NA.  Commercial lenses with high NA (or low F-number) can be purchased with reasonable prices.  A F1.4 DSLR lens is approximately $\$600$.  Detectors with large sensor size can be extremely expensive and difficult to manufacturer.  As a result, previous lens-coupled x-ray detector systems have been limited to small-scale imaging applications~\citep{lee2001, kim2005, tate2005, madden2006}.  
%
%
%
%
%
%
%
%
%
%We can derive the collection efficiency by first assuming that the lens can be treated as a thin lens.  This is when the distances $p$ and $q$ are measured using the front and rear principal planes, so we can still use the basic imaging equation, $p^{-1} + q^{-1} = f^{-1}$ and Eqn.~\ref{eqn:magnification}.  If the lens is used in conditions that do not satisfy paraxial approximation but are well corrected for spherical and coma aberrations, we can use the Abb$\acute{\mathrm{e}}$ sine condition to derive the collection efficiency.  Shown in Fig.~\ref{fig:Abbe}, the Abb$\acute{\mathrm{e}}$ sine condition uses spherical surfaces rather than principal planes.  Here the distance $p$ and $q$ are the radius of the spherical surface rather than the distance from the object to the principal planes.  The condition states that $p \, sin \theta = q \, sin \theta'$ even when paraxial approximation ($sin \theta \approx tan \theta \approx \theta $) does not hold true, which might be the case when imaging a large object with a lens that has a large NA.
%%
%\begin{figure}[h]
%\includegraphics[width = 6cm]{AbbeSineCondition.png}
%\caption{The Abb$\acute{\mathrm{e}}$ sine condition~\citep{Barrett2004}.}
%\label{fig:Abbe}
%\end{figure}
%
%The total solid angle collect by the lens is equal to
%%
%\begin{equation}
%\Omega_{lens} = \pi sin^2 \theta,
%\end{equation}
%%
%where $\theta$ is the marginal ray angle.  For a Lambertian point source with exitance equal to $\pi L_0$, the fraction of solid angle collected is $sin^2 \theta$.
%

%\subsection{X-ray Converters}
%\label{subsect:x-ray_converters}
%The x-ray digital radiography system is composed of two main components.  The first component is an x-ray converter that converts x-ray photons into secondary quanta.  The second component is a readout array that measures the output from the first component.  There are two general classes of x-ray converters: photoconductors and scintillators, shown in Fig.~\ref{fig:x-ray_detector}.
%
%\begin{figure}[ht]
%\centering
%	\begin{subfigure}[h]{0.3\linewidth}
%		\includegraphics[scale=1]{x-ray_detector_photoconductor.eps}
%		\caption{}
%		\label{fig:xray_photoconductor}
%	\end{subfigure}
%	\hspace{2 cm}
%	\begin{subfigure}[h]{0.3\linewidth}
%		\includegraphics[scale=1]{x-ray_detector_scintillator.eps}
%		\caption{}
%		\label{fig:xray_scintillator}
%	\end{subfigure}
%	\caption{Two types of x-ray converts, (a): a photoconductor, and (b): a scintillator.}
%	\label{fig:x-ray_detector}
%\end{figure}
%
%The first type of x-ray converters is a photoconductor, shown Fig.~\ref{fig:xray_photoconductor}.  The photoconductor absorbs the incident x-ray photon and converts it into an electron-hole pair.  The electron-hole pair drifts under an applied electric field, which is then collected by a capacitor inside each pixel.  The charge is read out via multiple electronic stages and converted into a digital signal.  Figure~\ref{fig:photoconductor_cross_section} shows a cross-section of a photoconductor.  The performance of an x-ray photoconductor is limited mainly by its x-ray sensitivity, noise, and image lag.  X-ray sensitivity refers primarily to the photoconductor's material used to interact with an x-ray.  Usually, a high atomic number, Z, will have a higher probability to interact with an x-ray, and hence higher x-ray sensitivity.  Another factor that affects the x-ray sensitivity is referred to as the $W$-value, which is proportional to the band gap energy in the material.  $W$-value is defined as the average energy required to create a single electron-hole pair.  Ideally, an extremely sensitive photoconductor should have a low $W$-value.  However, because a low $W$-value can also contribute to high thermal noise, there is a trade-off between x-ray sensitivity and noise.  Finally, image lag and ghosting in photoconductors are caused by incomplete charge collection and large charge fluctuations due to the trapping and detrapping of charge carriers by various traps or defects in the band gap.  These can be avoided or reduced by ensuring that the mean drift length of the generated charge carriers is greater than the thickness of the photoconductors~\citep{Kim2008, kasap2006}.  The most common material for photoconductors is amorphous selenium (a-Se) and is the most widely used material in commercial products.  Due to its low Z, a-Se is usually used in mammography devices operating at 20-30 kVp.  The low energy of the Se k-edge gives the material an absorption advantage over scintillator materials such as CsI(Tl) for the same material thickness~\citep{Yorkston2007}.  Other types of photoconductors, such as $\mathrm{PbI_2}$, $\mathrm{HgI_2}$, $\mathrm{PbO}$, $\mathrm{CdTe}$, and $\mathrm{CdZnTe}$, have been reported~\citep{springer2007}.  Photoconductors rely on electron-hole pair diffusion rather than optical photon diffusion, which means they do not have adequate time to diffuse laterally from their initial creation location before being collected.  As a result, photoconductors can achieve much higher resolution compared to scintillators.  The devices based on a-Se can provide spatial resolution that is close to the theoretical maximum predicted by a perfect Rect function response from the pixel~\citep{hunt5030}. 
%
%\begin{figure}[ht]
%\centering
%\includegraphics[scale = 0.5]{photoconductor_kasap2006.png}
%\caption{A cross-section of a photoconductor pixel~\citep{kasap2006}.}
%\label{fig:photoconductor_cross_section}
%\end{figure}
%
%The second type of x-ray converter is a scintillator, shown in Fig.~\ref{fig:xray_scintillator}.  Scintillators absorb x-ray photons and re-emit lower-energy photons typically in the visible range.  Traditional film-screen cassette systems use powdered phosphors screens such as gadolinium oxy-sulphide ($\mathrm{Gd_2O_2S \colon Tb}$) and place the screens on both sides of an x-ray film.  The light produced from the scintillator screens was then used to expose the film to create an image.  These traditional x-ray screens typically produce $\sim$1,000-2,000 visible photons per absorbed x-ray photon~\citep{trauernicht1988, trauernicht1990}. Newer types of scintillators such as columnar CsI(Tl) are grown as crystals in needle-like structures.  This allows these scintillators to be made much thicker, which increases the number of x-rays absorbed while maintaining the light spread in the scintillator compared to traditional powdered phosphor screens.  Figure~\ref{fig:scintillators} shows the scintillator structure of $\mathrm{Gd_2O_2S:Tb}$ vs. columnar CsI(Tl).
%
%\begin{figure}[ht]
%\includegraphics[width = 10cm]{scintilators.jpg}
%\caption{Two different types of scintillator structures viewed under SEM~\citep{scintillatorImage}}
%\label{fig:scintillators}
%\end{figure}
%
%%X-ray detectors that uses scintillators requires the scintillator to be thick enough to stop x-rays while limiting the amount of light spread in order to preserve spatial resolution.  The spatial resolution of a-Se based detectors does not depend on the photonconductor's thickness since the electron-hole pairs created inside the photoconductor does not drift laterally once it's created due to the biased voltage.  Rather, the detection efficiency is decreased as the a-Se layer increases.  This is because as the a-Se layer increases,  the charge carriers has more chance to be absorbed by the a-Se layer itself and decrease the number of charge carriers arriving at the charge collecting capacitor.
%%Since x-rays are ionizing radiation, we need to limit the exposure to the object, i.e. the human body, with as little radiation as possible while still be able to produce accurate diagnostic images.  This means that the x-ray converter needs to be able to absorb and convert as many incident x-ray photon as possible.  
%
%\subsection{Secondary Quantum Detectors}
%The most prevalent secondary quantum detectors are based on hydrogenated amorphous silicon (a-Si:H) arrays.  The fabrication technique is called plasma-enhanced chemical vapor deposition (PECVD), and its main commercial use is in the display market.  The fabrication process can routinely create devices as large as the body parts that are being imaged, up to $\sim$43 $\times$ 43 $\mathrm{cm^2}$.  This means that the detectors can have very high collection efficiencies with $\sim$50$\%$ and up to as high as $\sim$90$\%$~\citep{Yorkston2007}.
%Figure~\ref{fig:a-Si:H array} shows a general layout of the pixels on an a-Si:H array.  
%
%\begin{figure}[ht]
%\includegraphics[scale=0.5]{si_array_kim.png}
%\caption{The readout pixel array based on the a-Si:H photodiode~\citep{Kim2008}.}
%\label{fig:a-Si:H array}
%\end{figure}
%
%The sensing or storage element on the a-Si:H array depends on the type of x-ray converter used.  When used with a photoconductor, the storage element is a capacitor, where the a-Si:H array directly measures the electron-hole pairs that are created when x-ray photons are incident on the photoconductor.  When using the a-Si:H array with a scintillator, the storage element is a photodiode, which absorbs the visible light created by the scintillator after x-ray photons are absorbed. The photodiode then converts the visible light into electron-hole pairs.  As a result, the a-Si:H array made with photodiodes are often referred to as \textit{indirect} detectors because the photodiode is used as an intermediate step to convert visible light from the scintillator into electron-hole pairs.  The a-Si:H arrays made with photoconductors are referred to as \textit{direct} detectors because they directly measure the amount of electron-hole pairs created in the photoconductor by the a-Si:H pixel arrays.  In both cases, the final measurement is electrical charge.  
%
%Once electrical charges are accumulated on the storage element, the pixels are read out in a line-by-line mode by changing the control voltage on individual switch control lines.  Then, the individual pixels are read out through each data line.  The signal from each pixel is passed through peripheral readout electronics.  These readout electronics usually incorporate correlated double sampling and attempts to eliminate the 1/f noise and dc offset from the signal.  The signal also passes through a series of amplifiers and multiplexers and eventually to an analog-to-digital converter (ADC) before it is transferred to memory.  
%
%The leading manufacturer that uses a-Se technology is Hologic, which focuses mainly on mammography equipment, while GE HealthCare is the leading medical equipment company that pioneered the development of flat panel detectors (FPD) with CsI:Tl scintillators.  Most x-ray imaging detectors currently on the market are based on one of these two technologies.  
%
%\section{Lens-Coupled X-ray Detectors}
%The idea of using a lens-coupled camera system as an x-ray detector has been proposed and rejected before.  The basic problem comes down to photon collection efficiency.  While the phosphor screen must be at least the size of the object to be imaged, the lens must be able to capture the entire field of view and image the phosphor screen onto a CMOS or CCD detector with limited size.  From geometrical optics, we know that the magnification of a lens, $M = -q/p$, depends on the distance between the object to the lens $p$ and the distance from the lens to the image $q$. This same magnification is also the ratio between the size of the object and image.  Therefore, if the $M$ between the object and image is quite large, then the distance between the object to lens must be large as well.  This means that the lens must subtend a small solid angle in order to capture the entire field of view of the object, resulting in low photon collection efficiency.  The only way to overcome this problem is by having either a smaller field of view, a large detector, a faster lens, or all three.  As a result, previous lens-coupled x-ray detector systems have limited to small scale imaging applications~\citep{lee2001, kim2005, tate2005, madden2006}.  
%
%Recent sensor technology has improved tremendously, making it easier to purchase a camera with a large sensor size.  Current consumer-grade digital single-lens reflex (DSLR) cameras can be purchased with 36 mm $\times$ 24 mm detector size at around \$2000.  This improvement in sensor size allows us to decrease the distance between the lens and phosphor screen, therefore improving the photon collection efficiency.
%
%In this dissertation, two x-ray imaging systems, a digital radiography (DR) system and a computed tomography (CT) system that were built using the concept of lens-coupled detector system are introduced in chapter~\ref{chap:design_construction}.  A method of evaluating x-ray CT detectors using an observer model is presented in chapter~\ref{chap:model_observer}.   The x-ray CT system presented is a fully functioning image system complete with calibration and reconstruction algorithms.  These algorithms are explained in chapters~\ref{chap:calibration} and~\ref{chap:reconstruction}.  Finally, a few system designs for a potential brain imaging application are presented in chapter~\ref{chap:brain_imaging}.  
%
%\section{Notes}
%\comment{notes from the medical imaging hand book, chapter on flat-panel x-ray detectors}
%
%Indirect x-ray detection means that the image formation process is first converted from x-rays to visible photons, then finally to electrical charge.
%
%The active-matrix arrays used in the direct detection approach are ``\textit{similar to those used in display applications where charge on a capacitor plate controls the light transmission through the liquid crystal in AMLCDs}''.  It is called direct detection because the image formation is transferred from x-rays directly to electrical charge with no intermediate stage.
%
%The terms indirect or direct refers to the x-ray detection process where as the flat-panel array serve as an x-ray fluence detector rather than individual x-ray photon detector.
%
%Active-matrix arrays do not transfer signal from pixel to neighboring pixel but from the pixel element directly to the read out amplifier.  Each row of the active-matrix array is controlled separately by control voltage lines, and each column has separate read out amplifier.  As a result, the pixels are read out one horizontal line at a time.
%
%major disadvantage of the direct approach using a-Se is 1: it require a very high voltage to activate the a-Se layer.  If any manufacturing fault where to occur, this high voltage could damage the active-matrix array.  Also, a-Se has a low Z, and requires thick layers for high quantum efficiency at diagnostic energies (~100 keV).  
%
%an important advantage of a-Se is that it is uniform in imaging properties to a very fine scale (an amorphous material is entirely free from granularity), and can be easierly and cheaply made in large areas by a low-temperature process.
%
%$E_g$, energy gap of a photoconductor is ~2eV.  There are essentially no free carriers at room temperature.  another requirement is that carriers released by radiation have sufficient lifetime to reach the surface of the detection volume and be collected by the attached electrodes.  Such insulators are called photoconductors.
%
%W, the amount of energy necessary to create an electron-hole pair (e-h pair).  According to Klein~\citep{Klein1969}, on average, the energy of W is ~ 3$E_g$ to release an e-h pair.  Except a-Se, which has a high value of W (W = 42 eV.
%
%An important requirement of any x-ray photoconductor is that both holes and electrons should have Schubwegs ($S_h$ and $S_e$) that are much longer than the photoconductor thickness (T), that is, $S_h$ and $S_e$ >> T.  The Schubweb is the mean distance traversed by a charge carrier before it is trapped and given by the expression:
%\begin{equation}
%S = \mu \tau E,
%\end{equation}
%where $\mu$ is the drift mobility, $\tau$ is the carrier lifetime, and E is the applied field.  Typical applied field of E = 10 V $\mu m^{-1}$, which translate to $S_h$~6.5 to 65 mm and $S_e$ ~=.3 to 3 mm.  Typically a-Se is made to 0.2 to 1 mm, the importance of electron lifetime in controlling the x-ray photoconductivity can be appreciated.
%
%

























%
%The F-number describes the image-space cone of light for an object at infinity.  Under paraxial approximation, it is approximately equal to~\citep{greivenkampfieldguide},
%%
%\begin{equation}
%F \equiv \frac{f_E}{D_{EP}},
%\end{equation}
%%
%where $D_{EP}$ is the diameter of the entrance pupil and $f_E$ is the effective focal length.  For objects not at infinity, we can use the working F-number to describe the image-forming cone as,
%%
%\begin{equation}
%F_{w} \approx (1 + \lvert m \rvert) \, F.
%\end{equation}
%
%The marginal ray angle in image space can then be related to the working F-number as,
%%
%\begin{equation}
%sin \theta' = \frac{1}{2 F_{w}} = \frac{1}{2 F \, (1+ \lvert m \rvert)}.
%\end{equation}
%%
%Using Eqn.~\ref{eqn:magnification}, and the Abbe sine condition, we can derive the expression for the marginal ray angle in object space as,
%%
%\begin{equation}
%sin \theta = \frac{ \lvert m \rvert }{2 F \, (1+ \lvert m \rvert )}.
%\label{eqn:marginal_ray_angle}
%\end{equation}
%
%For a Lambertian surface emitter such as the x-ray phosphor screen, where the radiance, $L_0$, is completely independent of viewing direction, we find that the radiant exitance $M$ is equal to,
%%
%\begin{equation}
%\begin{aligned}
%M_{lamb} & = \int_{2\pi} L_0 \, cos \theta \, d\Omega = L_0 \int_0^{2 \pi} d\phi \int_0^{\pi/2} sin \theta \, cos \theta \, d\theta \\
%  & = \pi L_0.
%\label{eqn:exitance}
%\end{aligned}
%\end{equation}
%%
%where $d\Omega = sin \theta \, d\phi \, d\theta$ and is the total projected solid angle when integrated over $\theta$ and $\phi$.  For a non-Lambertian surface, the the radiance of the source depends on the viewing angle such that $L = L_0 \, cos \theta $.  Using similar approach as Eqn.~\ref{eqn:exitance}, the radiant exitance for a non-Lambertian source, such as columnar CsI, is
%%
%\begin{equation}
%\begin{aligned}
%M_{non-lamb} & = \int_{2 \pi} L \, cos\theta \, d\Omega = L_0 \int_0^{2 \pi} d\phi \int_0^{\pi/2} sin\theta \, cos^2 \theta \, d\theta \\
%    & = \pi L_0 \, \frac{2}{3}.
%\end{aligned}
%\end{equation}
%%
%Thus, we can see that the collection efficiency for a Lambertian source is $\pi sin^2 \theta$, and the collection efficiency for a non-Lambertian source is $\frac{2 \pi}{3} (1-cos^3 \theta)$.  For small angles where $sin^2 \theta \approx 1 - cos^3 \theta \approx \frac{3}{2}\theta^2$, the collection efficiency is the same for both types of sources, which after inserting Eqn.~\ref{eqn:marginal_ray_angle} equals to
%\begin{equation}
%\Omega = \frac{\pi \, m^2}{4 F^2 (1 + \lvert m \vert )^2}.
%\end{equation}
%However, when the collection angle increases, the discrepancy between the two sources increases.  Since $\theta$ is fairly small in both cases, the collection efficiencies are identical.