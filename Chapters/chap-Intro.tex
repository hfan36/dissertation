\chapter{Introduction}
\label{chap:intro}

Comment on how x-ray imaging is an important diagnostic tool in medical imaging. 
%Although both technology are used in the medical industry with great performance.  A way to lower the cost for x-ray imaging is by using a lens-coupled camera system to image the primary x-ray converted image onto a smaller detector.

\section{Review of X-ray Detectors}
\label{sect:review_x-ray_det}
Modern digital x-ray radiography systems can be divided into two general group using two different readout processes.  The first is based on storage phosphors where the x-ray image is first stored in an x-ray converter in a cassette form which later requires a separate mechanical readout process to record the image.  Typically, this separate readout process requires human intervention to transfer the storage phosphor cassette from the patient to the laser scanning station.  Systems that acquire images using this method are commonly known as Computed Radiography (CR) systems.  CR systems have been commercially available for almost two decades.  They are used for various applications and produce images with excellent image quality; they are not the focus of this dissertation.  For more formation, please refer to the review articles by Rowlands~\citep{Rowlands2002} and Kato~\citep{kato1994}, and the American Association of Physicists in Medicine (AAPM) Report N0. 93.~\citep{AAPM93}.  The second group of radiography system is an integrated system where the x-ray is detected and readout by the same device without any human intervention.  These are commonly known as Digital Radiography (DR) systems and they are the focus of this chapter.

\subsection{X-ray Converters}
\label{subsect:x-ray_converters}
The x-ray digital radiography system is composed of two main components.  The first component is an x-ray converter that converts x-ray photons into secondary quanta.  The second component is a read-out array that measures the output from the first component.  There are two general classes of x-ray converters, photoconductors and scintillators.  They are shown in Fig.~\ref{fig:x-ray_detector}.

\begin{figure}[ht]
\centering
\includegraphics[scale=1]{x-ray_detector}
\caption{There are two types of x-ray converts.  Left: a photoconductor, right: a scintillator}
\label{fig:x-ray_detector}
\end{figure}

The first type of x-ray converters is a photoconductor, shown on the left side in Fig.~\ref{fig:x-ray_detector}.  The photoconductor absorbs the incident x-ray photon and converts it into an electron-hole pair.  The electron-hole pair drifts under an applied electric field, which is then collected by a capacitor inside each pixel.  The charge is readout via multiple electronic stages and converted into a digital signal.  Figure~\ref{fig:photoconductor_cross_section} shows a cross-section of a photoconductor.  The performance of x-ray photoconductor is mainly limited by its x-ray sensitivity, noise, and image lag.  X-ray sensitivity refers mostly to the photoconductor's material to interact with x-ray.  Usually a high atomic number, Z, will have a higher probability to interact with x-ray, hence higher x-ray sensitivity.  Another factor that effects the x-ray sensitivity is referred to as $W$-value, which is proportional to the band gap energy in the material.  $W$-value is defined as the average energy required to create a single electron-hole pair.  Ideally an extremely sensitive photoconductor should have the low $W$-value.  However low $W$-value can also contribute to high thermal noise, so there is a trade off between x-ray sensitivity and noise.  Finally image lag and ghosting in photoconductors are caused by incomplete charge collection and large charge fluctuations due to the trapping and detrapping of charge carriers by various traps or defects in the band gap.  These can be avoided or reduced by making sure the mean drift length of the generated charge carriers greater than the thickness of the photoconductors~\citep{Kim2008, kasap2006}.  The most common material for photoconductors is amorphous Selenium (a-Se) and it is the most widely used material in commercial products.  Due to its low Z, a-Se is usually used in mammography devices operating at 20-30 kVp.  Since the low energy of the Se k-edge gives the material an absorption advantage over scintillator materials such as CsI(Tl) for the same material thickness~\citep{Yorkston2007}.  Other types of photoconductors such as $\mathrm{PbI_2}$, $\mathrm{HgI_2}$, $\mathrm{PbO}$, $\mathrm{CdTe}$, and $\mathrm{CdZnTe}$ have been reported~\citep{springer2007}.  Photoconductors rely on electron-hole pair diffusion rather than optical photon diffusion, which means they do not have adequate time to diffuse laterally from their initial creation location before being collected.  As a result photoconductors can achieve much higher resolution compared to scintillators.  The devices based on a-Se can provide spatial resolution that is close to the theoretical maximum predicted by a perfect Rect function response from the pixel~\citep{hunt5030}. 

\begin{figure}[ht]
\centering
\includegraphics[scale = 0.5]{photoconductor_kasap2006.png}
\caption{A cross section of a photoconductor pixel~\citep{kasap2006}.}
\label{fig:photoconductor_cross_section}
\end{figure}

The second type of x-ray converter is a scintillator, shown on the right side in Fig.~\ref{fig:x-ray_detector}.  Scintillators absorb x-ray photons and re-emit lower energy photons usually in the visible range.  Traditional film-screen cassette systems use powdered phosphors screens such as gadolinium oxy-sulphide ($\mathrm{Gd_2O_2S:Tb}$) and places the screens on both sides of an x-ray film.  The light produced from the scintillator screens were then used to expose the film to create an image.  These traditional x-ray screens typically produces $\sim$1000-2000 visible photons per absorbed x-ray photon~\citep{trauernicht1988, trauernicht1990}. Newer types of scintillators such as columnar CsI(Tl) are grown as a crystals in needle-like structures.  This allows these scintillators to be made much thicker, which increases the amount of x-ray absorbed while maintaining the light spread in the scintillator compared to traditional powered phosphor screens.  Figure~\ref{fig:scintillators} shows the scintillator structure of $\mathrm{Gd_2O_2S:Tb}$ vs. columnar CsI(Tl).

\begin{figure}[ht]
\includegraphics[width = 10cm]{scintilators.jpg}
\caption{Two different types of scintillator structures viewed under SEM~\citep{scintillatorImage}}
\label{fig:scintillators}
\end{figure}

%X-ray detectors that uses scintillators requires the scintillator to be thick enough to stop x-rays while limiting the amount of light spread in order to preserve spatial resolution.  The spatial resolution of a-Se based detectors does not depend on the photonconductor's thickness since the electron-hole pairs created inside the photoconductor does not drift laterally once it's created due to the biased voltage.  Rather, the detection efficiency is decreased as the a-Se layer increases.  This is because as the a-Se layer increases,  the charge carriers has more chance to be absorbed by the a-Se layer itself and decrease the number of charge carriers arriving at the charge collecting capacitor.
%Since x-rays are ionizing radiation, we need to limit the exposure to the object, i.e. the human body, with as little radiation as possible while still be able to produce accurate diagnostic images.  This means that the x-ray converter needs to be able to absorb and convert as many incident x-ray photon as possible.  

\subsection{Secondary Quantum Detectors}
The most prevalent secondary quantum detectors are based on hydrogenated amorphous silicon (a-Si:H) arrays.  The fabrication technique is called plasma enhanced chemical vapor deposition (PECVD) and its main commercial use is in the display market.  The fabrication process can routinely create devices as large as the body parts that are being imaged, up to $\sim$43 $\times$ 43 $cm^2$.  This means that the detectors can have very high collection efficiency with $\sim$50$\%$ and up to as high as $\sim$90$\%$ collection efficiency~\citep{Yorkston2007}.
Figure~\ref{fig:a-Si:H array} shows a general layout of the pixels on an a-Si:H array.  

\begin{figure}[ht]
\includegraphics[scale=0.5]{si_array_kim.png}
\caption{The readout pixel array based on the a-Si:H photodiode~\citep{Kim2008}}
\label{fig:a-Si:H array}
\end{figure}

The sensing or storage element on the a-Si:H array depends on the type of x-ray converter used.  When using with a photoconductor the storage element is a capacitor, where the a-Si:H array directly measures the electron-hole pairs that are created when x-ray photons are incident on the photoconductor.  When using a-Si:H array with a scintillator, the storage element is a photodiode and the photodiode absorbs the visible light created by the scintillator after x-ray photons are absorbed. The photodiode then converts the visible light into electron-hole pairs.  As a result, a-Si:H with photodiodes are often referred to as \textit{indirect} detectors because the photodiode is used as an intermediate step to convert visible light from the scintillator into electron-hole pairs.  a-Si:H made with photoconductors are referred to as \textit{direct} detector because it directly measures the amount of electron-hole pairs created in the photoconductor by the a-Si:H pixel arrays.  In both cases the final measurement is electrical charge.  

Once electrical charges are accumulated on the storage element, the pixels are read out in a line-by-line mode by changing the control voltage on individual switch control lines.  Then the individual pixels are read out through each data line.  The signal from each pixel is passed through peripheral readout electronics.  These readout electronics usually incorporates correlated double sampling and tries to eliminate the 1/f noise and dc offset from the signal.  The signal also passes through a series of amplifiers and multiplexers and eventually to an analog-to-digital converter (ADC) before it is transferred to memory.  

The leading manufacturer that uses a-Se technology is Hologic that focuses mainly on mammography equipment, while GE HealthCare is the leading medical equipment company that pioneered the development of flat panel detectors (FPD) with CsI:Tl scintillator.  Most x-ray imaging detectors out on the market are based one of these two technologies.  

\section{Lens-coupled X-ray Detectors}
The idea of using a lens-coupled camera system as x-ray detectors have been proposed and rejected before.  The basic problem comes down to photon collection efficiency.  While the phosphor screen must be at least the size of the object that needs to be imaged, the lens must be able to capture the entire field of view and image the phosphor screen onto a CMOS or CCD detector with limited size.  From geometrical optics we know that the magnification of a lens, $M = -q/p$, depends on the distance between the object to the lens $p$, and the distance from the lens to the image $q$. This same magnification is also the ratio between the size of the object and image.  So if the $M$ between the object and image is quite large, then the distance between the object to lens must be large as well.  This means that the lens must subtends small solid angle in order to capture the entire object field of view, resulting in low photon collection efficiency.  The only way to over come this problem is by having either a smaller field of view, or a large detector, or a faster lens, or all three.  As a result, previous lens-coupled x-ray detector systems have limited to small scale imaging applications~\citep{lee2001, kim2005, tate2005, madden2006}.  

Recent sensor technology has improved tremendously and it is quite easy to purchase a camera with a large sensor size.  Current consumer grade digital single-lens reflex (DSLR) cameras can be purchased at around \$2000 with 36 mm $\times$ 24 mm detector size.  This improvement in sensor size allows us to decrease the distance between the lens and phosphor screen, therefore improve the photon collection efficiency.

In this dissertation we introduce first two x-ray imaging systems in chapter~\ref{chap:design_construction}, a digital radiography (DR) system and a computed tomography (CT) system that were built using the concept of lens-coupled detector system.  We will present a method of evaluation x-ray CT detectors using an observer model in chapter~\ref{chap:model_observer}.   The x-ray CT system presented is a fully functioning image system complete with calibration and reconstruction algorithms.  These algorithms are explained in chapter~\ref{chap:calibration} and chapter~\ref{chap:reconstruction}.  Finally, we will present a few system designs for a potential brain imaging application in chapter~\ref{chap:brain_imaging}.  
