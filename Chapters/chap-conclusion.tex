\chapter{Conclusion}
The introduction chapter gave a brief overview of the two types of digital radiography systems used in the market today.  Our approach to the imaging problem was slightly different.  The motivation behind lens-coupled digital x-ray detector system was so it can be used in regions where access to modern medicine can be difficult.  We provided the equations to calculate the collection efficiency for lens-coupled system in chapter 1.  In chapter 2, we have demonstrated this idea by building a low-cost portable digital radiography unit (with special thank you to Brian Miller, Jared Moore, Stephen Moore, Heather Durko, and Lars Furenlid).  Pictures taken in Nepal using the hospital x-ray source was compared to images taken with our system and their the film-screen technique.  We have observed that the method used to develop x-ray films was less controlled and as a result, spots was observed in the x-ray film.  This is non-existent in the image acquired using our system.  We also noted that the DSLR camera has a large detector area and the number of photoelectrons created by a single x-ray photon is greater than the noise on the camera.  So in chapter 3, we set out to simulate a computed tomography system to calculated the performance of a lens-coupled detector system using the channelized-Hotelling observer.  We found that if we use the angle correlation information in the data to calculate signal detectability, it can be very useful to optimize the imaging system.  While the planar imaging system worked very well, we were interested to apply the same concept to a computed tomography system.  The construction of this system was shown in chapter 2.  It can be used to test the performance of the lens-coupled concept using different cameras and it is capable of acquiring projection images by rotating an object over 360 degrees.  Once the system was built and the hardware units were integrated in LabView, the calibration of the system was described in chapter 4.  An iterative reconstruction algorithm for the system was written using CUDA.  This reconstruction algorithm and results of the CT reconstruction using a DSLR camera were presented in chapter 5.  

We note that while we are constructing the CT system, both Nikon and Leica released newer cameras using large sensors that does not have color filtered.  The one by Nikon is a CMOS senor while the Leica uses a CCD.  The Nikon camera showed to have much higher quantum efficiency in the $\mathrm{Gd_2O_2S:Tb}$'s emission spectrum.  So it will be very interesting to test the camera and compare its performance with the Princeton and Andor camera.  It will also be interesting to test the performance of these cameras using columnar CsI instead of $\mathrm{Gd_2O_2S:Tb}$ to see if spatial resolution can be improved.  However access to a large size CsI will be difficult and/or expensive.  As to improvement that can be done to the computed tomography system setup.  I believe a better reconstruction algorithm and the ability to scan in helical orbit can be implemented.  Simple circular orbit used in this dissertation can be used to scan smaller objects as long as it is not truncated in any projection angle.  Finally it would be nice if the detective quantum efficiency of the detector system were measured and compared to other commercial DR systems.

