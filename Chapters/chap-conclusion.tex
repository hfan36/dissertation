\chapter{CONCLUSIONS AND FUTURE WORK}
The introduction chapter gave a brief overview of the two types of digital radiography systems used in the market today.  In both types of systems, the size of the detector panel is equal to the size of the x-ray image.  We have discussed a type of DR system where a lens is used to image an x-ray phosphor screen onto a digital camera.  In order to cover a large field of view, the lens demagnifies the visible-light image created by the phosphor onto the camera sensor.  One motivation behind the lens-coupled digital x-ray detector system was that it can be used in countries where access to modern medicine can be difficult.  We provided the equations to calculate the collection efficiency for the lens-coupled system in chapter 1.  

In chapter 2, we demonstrated this idea by building a low-cost portable digital radiography unit (with special thank you to Brian Miller, Jared Moore, Stephen Moore, Heather Durko, and Lars Furenlid).  Pictures taken in Nepal using the hospital x-ray source were compared to images taken with our system and their the film-screen technique.  We observed that the method used to develop x-ray films was less controlled in the Nepalese villages compared to the methods that were used in the US.  As a result, spots was observed in the x-ray film.  These spots are non-existent in the image acquired using our system.  

In chapter 3, we simulated a computed-tomography system to calculate the performance of a lens-coupled detector system using the channelized-Hotelling observer.  We found that the angular correlation information in the data can be used to calculate signal detectability.  This method can be very useful in optimizing an imaging system.  

In the planar imaging system, we have observed the effect of blur by the phosphor screen and the camera lens by measuring the system resolution and the noise power spectrum.  The noise power spectra were measured for the three different cameras described in chapter 2, and for three different magnifications.  These experiments were described in Appendix D.  In the near future, we would like to look at the effect of the blur by comparing the three cameras for a signal detection task with different number of incident x-ray photons and with different number of photoelectrons per x-ray photons.  We would also like to observe the effect of blur for a camera by comparing its performance in uniform and lumpy background for a range of x-ray photons.

The planar imaging system worked very well, and we were also interested in applying the same concept to a computed tomography (CT) system.  The construction of this system was shown in chapter 2.  This prototype CT system can be used to test the performance of the lens-coupled concept using different cameras.  It is capable of acquiring projection images by rotating an object over 360 degrees.  Once the system was built and the hardware units were integrated in LabView, the calibration of the system was described in chapter 4.  An iterative reconstruction algorithm for the system was written using CUDA.  This reconstruction algorithm and results of the CT reconstruction using a DSLR camera were presented in chapter 5.  

While we were constructing the CT system, both Nikon and Leica released newer cameras using large sensors that do not have color filters.  The one by Nikon is a CMOS senor while the Leica uses a CCD.  The specification of the Nikon camera indicated that it has a much higher quantum efficiency in the $\mathrm{Gd_2O_2S}$:Tb's emission spectrum.  It will be very interesting to test the camera and compare its performance with the Princeton and Andor camera.  It will also be interesting to test the performance of these cameras using columnar CsI instead of $\mathrm{Gd_2O_2S}$:Tb to see if spatial resolution can be improved.  However a large CsI will be expensive.  The ability to scan in helical orbit can be implemented to improve the CT system.  Finally the detective quantum efficiency of the detector system can be measured and compared to other commercial DR systems.

