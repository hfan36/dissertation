\chapter{Application in brain imaging}
\section{Background}

\noindent \textbf{Patient positioning} \\
\noindent According to the AAPM (American Association of Physicists in Medicine), patient positioning should follow the below list\citep{aapm_headCT2012}
\begin{itemize}
\item patient should lie face upward with the head in a head-holder device whenever possible.  The patient should enter the gantry head first.
\item the scan angle should be parallel to a line created by the supraorbital ridge and the inner table of the posterior margin of the foramen magnum.  This may be accomplished by either tilting the patient's chin toward the chest ("tucked" position" or tilting the gantry.
\end{itemize}

\noindent \textbf{scan range} \\
\noindent Top of C1 lamina through top of calvarium \\

\subsection{x-ray source requirements}
\subsection{phosphor screen criteria}

\begin{itemize}
\item type of phosphor screen, $Gd_2O_2S:Tb$ vs others: general can be chosen by emission spectra or energy range, for example, for high energy (voltage), use $Gd_2O_2S:Tb$, for low voltage one can use $La_2O_2S:Tb$ \cite{Kandarakis2001}

\item grain size of the phosphor, which relate to light conversion efficiency and attenuation coefficients.

\end{itemize}

X-ray absorption in radiographic phosphor directly determines both the relative brightness and magnitude of the signal fluctuation referred to as ``quantum mottle'', or ``scintillation noise''.\cite{swank1973}

In an x-ray imaging system that uses phosphor, the detector x-ray pulses are usually integrated.  These pulses in general does not have an uniform size, and are distributed acoording to some probability distribution.  Three major factors that contribute to this distribution:

\begin{itemize}
\item the incident x-ray energy distribution
\item the absorbed energy distribution which results from various absorption processes
\item the light output distribution which results from unequal light propagation from different parts of the phosphor to the detector.
\end{itemize}


Emission wavelength is approximately 545 nm. \cite{Berzins1983}, \cite{Moy1993}, \cite{shepherd1995}, \cite{blasse1994}

$Gd_2O_2S:Tb$ is not susceptible to long-term radiation damage \cite{Antonuk1990}

Swank reported a value for 0.4 ms for the decay time in measurements using Am-241 (59.6 keV gamma rays) \cite{Swank1974}

\subsection{detector selection}



\section{Measurements}

